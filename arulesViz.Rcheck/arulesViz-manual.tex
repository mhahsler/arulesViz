\nonstopmode{}
\documentclass[letterpaper]{book}
\usepackage[times,inconsolata,hyper]{Rd}
\usepackage{makeidx}
\usepackage[utf8,latin1]{inputenc}
% \usepackage{graphicx} % @USE GRAPHICX@
\makeindex{}
\begin{document}
\chapter*{}
\begin{center}
{\textbf{\huge Package `arulesViz'}}
\par\bigskip{\large \today}
\end{center}
\begin{description}
\raggedright{}
\item[Version]\AsIs{1.3-0.2}
\item[Date]\AsIs{2017-12-18}
\item[Title]\AsIs{Visualizing Association Rules and Frequent Itemsets}
\item[Depends]\AsIs{arules (>= 1.4.1), grid}
\item[Imports]\AsIs{scatterplot3d, vcd, seriation, igraph (>= 1.0.0), graphics,
methods, utils, grDevices, stats, colorspace, DT, plotly,
visNetwork}
\item[Suggests]\AsIs{shiny}
\item[Description]\AsIs{Extends package 'arules' with various visualization techniques for association rules and itemsets. The package also includes several interactive visualizations for rule exploration.}
\item[License]\AsIs{GPL-3}
\item[URL]\AsIs{}\url{https://github.com/mhahsler/arulesViz}\AsIs{,
}\url{http://lyle.smu.edu/IDA/arules/}\AsIs{}
\item[BugReports]\AsIs{}\url{https://github.com/mhahsler/arulesViz}\AsIs{}
\item[Copyright]\AsIs{(C) 2011 Michael Hahsler and Sudheer Chelluboina}
\item[RoxygenNote]\AsIs{6.0.1}
\item[NeedsCompilation]\AsIs{no}
\item[Author]\AsIs{Michael Hahsler [aut, cre, cph],
Sudheer Chelluboina [ctb]}
\item[Maintainer]\AsIs{Michael Hahsler }\email{mhahsler@lyle.smu.edu}\AsIs{}
\end{description}
\Rdcontents{\R{} topics documented:}
\inputencoding{utf8}
\HeaderA{inspectDT}{Inspect Associations Interactively Using datatable}{inspectDT}
\aliasA{datatable}{inspectDT}{datatable}
\aliasA{inspect}{inspectDT}{inspect}
\keyword{print}{inspectDT}
%
\begin{Description}\relax
Uses \pkg{datatable} to create a HTML table widget using the DataTables 
library. Rules can be interactively filtered and sorted.
\end{Description}
%
\begin{Usage}
\begin{verbatim}
inspectDT(x, ...) 
\end{verbatim}
\end{Usage}
%
\begin{Arguments}
\begin{ldescription}
\item[\code{x}]  an object of class "rules" or "itemsets". 
\item[\code{...}]  additional arguments. \code{precision} controls the precision
used to print the quality measures (defaults to 2). All other arguments 
are passed on to \code{datatable}.
\end{ldescription}
\end{Arguments}
%
\begin{Value}
A datatable htmlwidget.
\end{Value}
%
\begin{Author}\relax
Michael Hahsler
\end{Author}
%
\begin{SeeAlso}\relax
\code{\LinkA{datatable}{datatable}} in \pkg{DT}.
\end{SeeAlso}
%
\begin{Examples}
\begin{ExampleCode}
## Not run: 
data(Groceries)
rules <- apriori(Groceries, parameter=list(support=0.005, confidence=0.5))
rules

inspectDT(rules)

### save table as a html page.
p <- inspectDT(rules)
htmlwidgets::saveWidget(p, "arules.html", selfcontained = FALSE)
browseURL("arules.html")
## End(Not run)
\end{ExampleCode}
\end{Examples}
\inputencoding{utf8}
\HeaderA{plot}{Visualize Association Rules and Itemsets}{plot}
\methaliasA{plot.grouped\_matrix}{plot}{plot.grouped.Rul.matrix}
\methaliasA{plot.itemsets}{plot}{plot.itemsets}
\methaliasA{plot.rules}{plot}{plot.rules}
\keyword{hplot}{plot}
%
\begin{Description}\relax
Methods (S3) to visualize association rules and itemsets.
Implemented are several popular visualization methods 
including scatter plots with shading (two-key plots), 
graph based visualizations, doubledecker plots, etc.

Many plots can use different rendering engines including
static standard plots (mostly using \pkg{grid}), 
standard plots with interactive manipulation 
and interactive HTML widget-based visualizations.
\end{Description}
%
\begin{Usage}
\begin{verbatim}
## S3 method for class 'rules'
plot(x, method = NULL, measure = "support", shading = "lift", 
    interactive = NULL, engine = "default", data = NULL, control = NULL, ...)
## S3 method for class 'itemsets'
plot(x, method = NULL, measure = "support", shading = NA,
    interactive = NULL, engine = "default", data = NULL, control = NULL, ...)
\end{verbatim}
\end{Usage}
%
\begin{Arguments}
\begin{ldescription}
\item[\code{x}]  an object of class "rules" or "itemsets". 
\item[\code{method}]  a string with value "scatterplot", "two-key plot", "matrix", 
"matrix3D",  "mosaic", 
"doubledecker", "graph", "paracoord" or "grouped", "iplots" selecting the 
visualization method (see Details).

\item[\code{measure}]  measure(s) of interestingness 
(e.g., "support", "confidence", "lift", "order") used in the visualization. Some 
visualization methods need one measure, others take a vector with two 
measures (e.g., scatterplot). In some plots (e.g., graphs) \code{NA} 
can be used to suppress using a measure. 

\item[\code{shading}]  measure of interestingness used
for the color of the points/arrows/nodes
(e.g., "support", "confidence", "lift"). The default is "lift".
\code{NA} can be often used to suppress shading.

\item[\code{interactive}]  deprecated. See parameter \code{engine} below. 

\item[\code{control}]  a list of control parameters for the plot. The available
control parameters depend on the used visualization method (see Details).

\item[\code{data}]   the dataset (class "transactions") 
used to generate the rules/itemsets. Only 
"mosaic" and "doubledecker" require the original data.

\item[\code{engine}]  a string indicating the plotting engine used to 
render the plot. 
The "default" engine uses (mostly) \pkg{grid}, 
but some plots can produce interactive 
interactive grid visualizations using engine "interactive", or
HTML widgets using engine 
"htmlwidget". These widgets can be saved as stand-alone web pages 
(see Examples). Note that HTML widgets tend to get very slow 
or unresponsive for 
too many rules. To prevent this situation, the control parameter 
\code{max} sets a limit, and the user is warned if the limit is reached. 


\item[\code{...}]  Further arguments are added for convenience to the \code{control} list.
\end{ldescription}
\end{Arguments}
%
\begin{Details}\relax
Most visualization techniques are described by Bruzzese and Davino (2008),
however, we added more color shading, reordering and interactive features.
Many visualization methods take extra parameters as the \code{control} parameter list. Although, we have tried to keep control parameters consistent, the available control parameters vary (slightly) from visualization method to visualization method. A complete list of parameters with default
values can be obtained using verbose mode. For example, 

\code{plot(rules, method = "graph", control = list(verbose = TRUE))} 

prints a complete list of control parameters for method "graph" 
(default engine).

The following visualization method are available:

\begin{description}

\item["scatterplot", "two-key plot"]  
This visualization method draws a two dimensional scatterplot with different
measures of interestingness (parameter "measure") on the axes and a third 
measure (parameter "shading") is represented by the color of the points. 
There is a special value for shading called "order" which produces a
two-key plot where the color of the points represents the length (order) 
of the rule.

Interactive manipulations are available. 
Engine "htmlwidget" is available to produce an interactive web-based
visualization (uses \pkg{plotly}).


\item["matrix"]  
Arranges the association rules as a matrix with the itemsets in the antecedents
on one axis and the itemsets in the consequents on the other.  The
measure of interestingness (first element of \code{measure}) is either visualized by a color (darker means a higher value for the
measure) or as the height of a bar (engine "3d"). Interactive
visualizations using engine "interactive" or "htmlwidget" (via \pkg{plotly})
are available.


\item["grouped matrix"] 
Grouped matrix-based visualization (Hahsler and Karpienko, 2016; Hahsler 2016). 
Antecedents (columns) in the matrix are
grouped using clustering. Groups are represented by the most
interesting item (highest ratio of support in the group to support in all rules) 
in the group. Balloons
in the matrix are used to represent with what consequent the antecedents are 
connected.

Interactive manipulations (zooming into groups and identifying rules) are available. 

The list of control parameters for this method includes:
\begin{description}

\item["main"] plot title
\item["k"] number of antecedent groups (default: 20)
\item["rhs\_max"] maximal number of RHSs to show. The rest are 
suppressed. (default: 10)
\item["lhs\_items"] number of LHS items shown (default: 2)
\item["aggr.fun"] aggregation function
can be any function computing a scalar from a vector
(e.g., min, mean (default), median, sum, max). It is also used
to reorder the balloons in the plot.
\item["col"] color palette (default is 100 heat colors.)
\item["gp\_labels", "gp\_main", "gp\_labs", "gp\_lines"] \code{gpar()} objects used to specify color, font and font size for
different elements.

\end{description}



\item["graph"] 
Represents the rules (or itemsets) as a graph with items, itemsets, and
rules represented as vertices.

Several engines are available. The default engine uses \pkg{igraph} (\code{plot.igraph} and \code{tkplot} for the interactive visualization). 
\code{...} arguments are passed on to the respective plotting function (use for color, etc.).

Alternatively, the engines "graphviz" (\pkg{Rgraphviz}) and "htmlwidget" (\pkg{visNetwork}) are available.
Note that Rgraphviz has to be installed separately from 
\url{http://www.bioconductor.org/}.


\item["doubledecker", "mosaic"] 
Represents a single rule as a doubledecker or mosaic plot.
Parameter \code{data} has to be specified to compute the needed contingency
table. No interactive version is available.


\item["paracoord"] 
Represents the rules (or itemsets) as a parallel coordinate plot.
Currently there is no interactive version available.


\item["iplots"] 
Experimental interactive plots (package \pkg{iplots}) 
which support selection, highlighting, 
brushing, etc. Currently plots a scatterplot (support vs. confidence) and
several histograms. Interactive manipulations are available. 



\end{description}

\end{Details}
%
\begin{Value}
Several interactive plots return a set of selected rules/itemsets. Other plots
might return other data structures. For example, graph-based
plots return the graph (invisibly). Engine "htmlwidget" always returns an object of class htmlwidget. 
\end{Value}
%
\begin{Author}\relax
Michael Hahsler and Sudheer Chelluboina. Some visualizations are based on 
the implementation by Martin Vodenicharov.
\end{Author}
%
\begin{References}\relax
Bruzzese, D. and Davino, C. (2008), Visual Mining of Association Rules, in
Visual Data Mining: Theory, Techniques and Tools for Visual Analytics,
Springer-Verlag, pp. 103--122.

Hahsler, M. and Karpienko, R. (2016) Visualizing Association Rules in Hierarchical Groups. \emph{Journal of Business Economics,} accepted for publication, 2016.

Hahsler, M. (2016) Grouping association rules using lift. In C. Iyigun, R. Moghaddess, and A. Oztekin, editors, 11th INFORMS Workshop on Data Mining and Decision Analytics (DM-DA 2016).
\end{References}
%
\begin{SeeAlso}\relax
\code{\LinkA{plotly\_arules}{plotly.Rul.arules}},
\code{\LinkA{scatterplot3d}{scatterplot3d}} in \pkg{scatterplot3d},
\code{\LinkA{plot.igraph}{plot.igraph}} and
\code{\LinkA{tkplot}{tkplot}} in \pkg{igraph},
\code{\LinkA{seriate}{seriate}} in \pkg{seriation}
\end{SeeAlso}
%
\begin{Examples}
\begin{ExampleCode}
data(Groceries)
rules <- apriori(Groceries, parameter=list(support=0.001, confidence=0.8))
rules

## Scatterplot
## -----------
plot(rules)

## Scatterplot with custom colors
library(colorspace) # for sequential_hcl
plot(rules, control = list(col=sequential_hcl(100)))
plot(rules, col=sequential_hcl(100))
plot(rules, col=grey.colors(50, alpha =.8))

## See all control options using verbose
plot(rules, verbose = TRUE)

## Interactive plot (selected rules are returned)
## Not run: 
sel <- plot(rules, engine = "interactive")
## End(Not run)

## Create a html widget for interactive visualization
## Not run: 
plot(rules, engine = "htmlwidget")
## End(Not run)

## Two-key plot (is a scatterplot with shading = "order")
plot(rules, method = "two-key plot")

  
## Matrix shading
## --------------

## The following techniques work better with fewer rules
subrules <- subset(rules, lift>5)
subrules

## 2D matrix with shading
plot(subrules, method="matrix")

## 3D matrix
plot(subrules, method="matrix", engine = "3d")

## Matrix with two measures
plot(subrules, method="matrix", shading=c("lift", "confidence"))

## Interactive matrix plot (default interactive and as a html widget)
## Not run: 
plot(subrules, method="matrix", engine="interactive")
plot(subrules, method="matrix", engine="htmlwidget")
## End(Not run)

## Grouped matrix plot
## -------------------

plot(rules, method="grouped matrix")
plot(rules, method="grouped matrix", 
  col = grey.colors(10), 
  gp_labels = gpar(col = "blue", cex=1, fontface="italic"))

## Interactive grouped matrix plot
## Not run: 
sel <- plot(rules, method="grouped", engine = "interactive")
## End(Not run)

## Graphs
## ------

## Graphs only work well with very few rules
subrules2 <- sample(subrules, 25)

plot(subrules2, method="graph")

## Custom colors
plot(subrules2, method="graph", 
  nodeCol = grey.colors(10), edgeCol = grey(.7), alpha = 1)

## igraph layout generators can be used (see ? igraph::layout_)
plot(subrules2, method="graph", layout=igraph::in_circle())
plot(subrules2, method="graph", 
  layout=igraph::with_graphopt(spring.const=5, mass=50))

## Graph rendering using Graphviz
## Not run: 
plot(subrules2, method="graph", engine="graphviz")
## End(Not run)

## Default interactive plot (using igraph's tkplot)
## Not run: 
plot(subrules2, method="graph", engine = "interactive")
## End(Not run)

## Interactive graph as a html widget (using igraph layout)
## Not run: 
plot(subrules2, method="graph", engine="htmlwidget")
plot(subrules2, method="graph", engine="htmlwidget", 
  igraphLayout = "layout_in_circle")

## End(Not run)

## Parallel coordinates plot
## -------------------------

plot(subrules2, method="paracoord")
plot(subrules2, method="paracoord", reorder=TRUE)

## Doubledecker plot 
## -----------------

## Note: only works for a single rule
oneRule <- sample(rules, 1)
inspect(oneRule)
plot(oneRule, method="doubledecker", data = Groceries)

## Itemsets
## --------

itemsets <- eclat(Groceries, parameter = list(support = 0.02, minlen=2))
plot(itemsets)
plot(itemsets, method="graph")
plot(itemsets, method="paracoord", alpha=.5, reorder=TRUE)

## Add more quality measures to use for the scatterplot
## ----------------------------------------------------

quality(itemsets) <- interestMeasure(itemsets, trans=Groceries)
head(quality(itemsets))
plot(itemsets, measure=c("support", "allConfidence"), shading="lift")

## Save HTML widget as web page
## ----------------------------
## Not run: 
p <- plot(rules, engine = "html")
htmlwidgets::saveWidget(p, "arules.html", selfcontained = FALSE)
browseURL("arules.html")
## End(Not run)
# Note: selfcontained seems to make the browser slow.
\end{ExampleCode}
\end{Examples}
\inputencoding{utf8}
\HeaderA{plotly\_arules}{Interactive Scatter Plot for Association Rules using plotly}{plotly.Rul.arules}
\aliasA{plotly}{plotly\_arules}{plotly}
\keyword{hplot}{plotly\_arules}
%
\begin{Description}\relax
Plot an interactive scatter plot for association rules using \pkg{plotly}.
\end{Description}
%
\begin{Usage}
\begin{verbatim}
plotly_arules(x, method = "scatterplot", measure = c("support", "confidence"), 
  shading = "lift", max = 1000, ...)
\end{verbatim}
\end{Usage}
%
\begin{Arguments}
\begin{ldescription}
\item[\code{x}]  an object of class "rules". 
\item[\code{method}]  currently the methods "scatterplot", "two-key plot"
and "matrix" are supported.
\item[\code{measure}]  measure(s) of interestingness 
(e.g., "support", "confidence", "lift", "order") used in the visualization as x
and y-axis.
\item[\code{shading}]  measure of interestingness used for color shading.
\item[\code{max}]  client side processing in plotly is expensive. We restrict the 
number of rules to the max best rules (according to the measure used for shading.)
\item[\code{...}]  The following additional arguments can be used: \code{colors} to specify a color palette, \code{precision}
to specify the precision used for printing quality measures,
\code{jitter} to reduce overplotting in scatterplots 
(defaults to .1 if overplotting would occur), and \code{marker} with 
a list of markter attributes (e.g., size, symbol and opacity).
\end{ldescription}
\end{Arguments}
%
\begin{Value}
The plotly object (plotly\_hash) which can be saved as a htmlwidget.
\end{Value}
%
\begin{Examples}
\begin{ExampleCode}
## Not run:  
library(plotly)
data(Groceries)
rules <- apriori(Groceries, parameter=list(support=0.001, confidence=0.8))
rules

# interactive scatter plot visualization
plotly_arules(rules)
plotly_arules(rules, measure = c("support", "lift"), shading = "confidence")
plotly_arules(rules, method = "two-key plot")

# add jitter, change color and markers and add a title
plotly_arules(rules, jitter = 10, 
  marker = list(opacity = .7, size = 10, symbol = 1), 
  colors = c("blue", "green")) 


# save a plot as a html page
p <- plotly_arules(rules)
htmlwidgets::saveWidget(p, "arules.html", selfcontained = FALSE)
browseURL("arules.html")
# Note: selfcontained seems to make the browser slow.

# interactive matrix visualization
plotly_arules(rules, method = "matrix") 

## End(Not run)
\end{ExampleCode}
\end{Examples}
\inputencoding{utf8}
\HeaderA{saveAsGraph}{Save rules or itemsets as a graph description}{saveAsGraph}
\keyword{file}{saveAsGraph}
%
\begin{Description}\relax
This function 
saves a rule sat as a graph description in different formats 
(e.g., GraphML, dimacs, dot).
\end{Description}
%
\begin{Usage}
\begin{verbatim}
saveAsGraph(x, file, type="items", format="graphml")
\end{verbatim}
\end{Usage}
%
\begin{Arguments}
\begin{ldescription}
\item[\code{x}]  an object of class "rules" or "itemsets".
\item[\code{file}]  file name
\item[\code{type}]  see type in plot with method "graph" 
(e.g., "itemsets", "items").
\item[\code{format}]  file format (e.g., "edgelist", 
"graphml", "dimacs", "gml", "dot"). See \code{write.graph} in package 
\pkg{igraph}.
\end{ldescription}
\end{Arguments}
%
\begin{Author}\relax
Michael Hahsler
\end{Author}
%
\begin{SeeAlso}\relax
\code{\LinkA{plot}{plot}}, \code{\LinkA{write.graph}{write.graph}} in \pkg{igraph}
\end{SeeAlso}
%
\begin{Examples}
\begin{ExampleCode}
data("Groceries")
rules <- apriori(Groceries, parameter=list(support=0.01, confidence=0.5))

saveAsGraph(rules, "rules.graphml")

## clean up
unlink("rules.graphml")
\end{ExampleCode}
\end{Examples}
\inputencoding{utf8}
\HeaderA{shiny}{Visualize Association Rules and Itemsets with Shiny}{shiny}
\aliasA{shiny\_arules}{shiny}{shiny.Rul.arules}
%
\begin{Description}\relax
Visualize assocation rules and itemsets using \pkg{shiny}.
\end{Description}
%
\begin{Usage}
\begin{verbatim}
shiny_arules(dataset, vars=0, supp=0.1, conf=0.5, lift = 0)
\end{verbatim}
\end{Usage}
%
\begin{Arguments}
\begin{ldescription}
\item[\code{dataset}] Base transactions object from which rules will be mined and visualized. Also takes a rules object containing already mined rules.
\item[\code{vars}] Default number of variables to include in initial rule mining. If a value is not passed to this parameter, all available variables will be used.
\item[\code{supp}] Default minimum support threshold for rule mining
\item[\code{conf}] Default minimum confidence threshold for rule mining
\item[\code{lift}] Default minimum lift threshold for displayed rules
\end{ldescription}
\end{Arguments}
%
\begin{Author}\relax
Tyler Giallanza. Adapted from functions originally created by Andrew Brooks.
See https://github.com/brooksandrew/Rsenal for the original code.
\end{Author}
%
\begin{Examples}
\begin{ExampleCode}
data(Adult)
## Not run: 
  shiny_arules(Adult)

## End(Not run)
\end{ExampleCode}
\end{Examples}
\printindex{}
\end{document}
